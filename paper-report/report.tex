\documentclass[11pt]{article}
\usepackage[utf8]{inputenc}
\usepackage{amsmath, amsthm, amssymb}
\usepackage{graphicx}
\author{Fabian Alenius, Kjell Winblad and Chongyang Sun} 
\title{Summary of Modeling Customer Relationships as Markov Chains}
\begin{document}
\maketitle


\section{Introduction}
The lifetime value (LTV) of a customer is the net present value of the cash flows attributed to the relationship with a customer.
Phillip E. Pfeifer and Robert L. Carraway \cite{customer} explains how the LTV concept can be used together with Markov Chains Models (MCM) to make managerial decisions on an individual customer basis.
The advantage MCM has over traditional methods include its flexibility and that it is a probabilistic model.
A probabilistic model explicitly accounts for the uncertainty surrounding customer relationships.
It also allows the use of language of probability and expected value when reasoning about a firm's future relationship with an individual customer.

In marketing, the Recency Frequency Monetary (RFM) framework is commonly used and it fits nicely with MCM.
Recency describes how long ago the customer purchased something.
Frequency describes how many times the customer has bought something from the firm.
Monetary value describes the average value of the customer purchase.
These concepts are used to define the states in the MCM.

The MCM consists of the following vectors and matrixes: 
\begin{enumerate}
\item The one-step transition matrix \textbf{P} defines the state transition probabilities.
\item The reward vector \textbf{R}  describes the reward received in each state. This value is a function of customer purchases and expenditures.
\item The value vector \textbf{V} describes  the expected present value of a customer in a specific sate. If the value is negative then that customer is effectively a loss for the company.
\end{enumerate}

Equation \ref{eq1} shows how the value vector \textbf{V} is defined in terms of the transition matrix  \textbf{P} and reward vector  \textbf{R} with a finite time horizon T.
The discount factor $d$ sets how much future rewards are discounted when calculating the present value.
\begin{equation}\label{eq1}
\textbf{V}^T = \sum_{t=0}^T  [(1 + d)^{-1} \textbf{P}]^t \cdot \textbf{R}
\end{equation}

When considering an infinite time horizon, the equation changes to Equation \ref{eq2}.
\begin{equation}\label{eq2}
\textbf{V} \equiv \lim_{T \rightarrow \infty} \textbf{V}^T = \{\textbf{I} - (1 + d)^{-1} \cdot \textbf{P} \}^{-1} \cdot \textbf{R}
\end{equation}


Therefore, given the economic and probabilistic assumptions of the model, the firm can evaluate the expected present values from $\textbf{V}$, such as, the last recency with the negtive value in the vector $\textbf{V}$ telling that the firm can do relatively better when curtailing its relationships with this customer after such recency.

According to the above observation, one can modify the Markov decision model with ease and reformulate the model easily by using less states and setting $\textbf{p}_t$ with negative values equal to zeros and the corresponding elements $\textbf{r}_t$ in $\textbf{R}$ equal to zeros as well.

Futhermore, whenever purchase probabilities, remarketing expenditures and net contributtions denpending on recency are considered in the model, MCM can be ajusted to it by expanding the state space and breaking out the original recency 1 state into several new states corresponding to the listed affecting factors depending on customer recency at the time of purchase. 

The MCM can not only be used for customer migration situations, but also for customer retention situations. For customer retention, the firm could start a fairly simple model by assigning MCM to three states:  Prospect, Customer, Former Customer. 

The last case discussed in the paper illustrates how to apply MCM to a situation where the firm believes that purchase probabilities, net contribution, and remarketing expenditures all depend on the recency, frequency and monetary value of past purchases.The MCM uses states defined by $\textbf{(r, f, m)}$ where the elements are integers with some upper bounds. Particularly, monetary values categories are paid much attention and single last purchase amount is chosen to avoid its non-Markovian nature instead of moving average.




\section{Discussion}
The authors main points with the paper seem to be to show that MCMs can be used to model many different customer relationship scenarios. They do that by showing some examples of scenarios that can be modeled with MCM. They also show with examples how the models can be used to change a company's relationship policy. They argue that MCM can be very useful due to it's flexibility. This is clearly illustrated by the examples. For example MCMs can be used to model customer retention and customer migration scenarios. One advantage of using MCM for customer relationship problems seem to be that the MCM naturally account for the stochasticity in customer relationships.

The authors do not write anything about the validity of the assumptions underlying the models. 
For example, the time period for purchases are assumed to be the same and fixed.

\bibliographystyle{plain}	% (uses file "plain.bst")
\bibliography{myrefs}		% expects file "myrefs.bib"
\end{document}