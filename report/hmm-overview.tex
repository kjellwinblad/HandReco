
For pattern recognition problem, like handwritten image recognition, there has  always been some randomness and uncertainty from the source recognition data. Stocastic modeling is known to deal with these problem efficiently by using probabilistic models \cite{Cho1995}.  Among such stochastic approaches, Hidden Markov Models have been widely used to model dynamic signals.
The Hidden Markov Model treats the data as a sequence of observations, while consisting hidden states connected to each other by transition probabilities.
 
An HMM is characterized by the following\cite{Rabiner1989}:
\begin{enumerate}
\item	N, the number of states in the model.
\item	M, the number of distinct observation symbols per state.
\item	A, the transition probability distribution.
\item	B, the observation symbol probability distribution in state
\item	Pi, the initial state distribution.
\end{enumerate}

In contrast to knowledge-based approach, HMMs use statistical algorithms that can automatically extract knowledge from the samples. Also, HMMs model the pattern implicitly by different paths in the stochastic work. The strength of modeling power can be enhanced by showing more samples
\cite{Cho1995}.
