Recognition of handwritten characters has been a popular research area for years because it can be used in many different applications.
Formally, handwritten recognition is the task of transforming a language represented in graphical form into its symbolic representation \cite{introsurvey}.
There are two different approaches to handwritten recognition, \textit{online} and \textit{offline}.
In the online approach we know the order in which the strokes and individual points were drawn, for example if the characters were recorded by a digital pen.
In the offline approach we are only given the final image.
Online recognition is primarily used for signature verification, author authentication and digital pens.
The potential application areas for offline recognition include postal automation, bank cheque processing and automatic data entry \cite{intro1}.
In this paper we only consider offline handwritten recognition.

The ultimate goal in handwritten recognition is to recognize words, however one way to potentially decompose or simplify the problem is to segment words it into individual characters. \cite{intro-Yacoubi}
Segmentation can either be done explicitly or implicitly.
Despite impressive progress over the last couple of decades, performance is still far away from human performance.
\cite{Cheriet}
