\documentclass[11pt]{article}
\usepackage[utf8]{inputenc}
\usepackage{amsmath, amsthm, amssymb}
\usepackage{graphicx}
\author{Fabian Alenius, Kjell Winblad and Chongyang Sun} \title{Handwritten Character Recgonition}
\begin{document}
\maketitle

\begin{abstract}
Hello.

\end{abstract}

\section{Introduction}
%1. Give general introduction to area.
%2. Describe why it's important.
%3. Applications.
%4. Contributions.

\cite{trec}
%Fabian
\section{Introduction}

Recognition of handwritten text has been a popular research area for decades because it can be used in so many different applications.
There are two different approaches to handwritten recognition, \textit{online} and \textit{offline}.
In the online approach we know the order in which the strokes and individual points were drawn.
This information can easily be captured if the text is recorded by a digital pen or on a touchscreen.
In the offline approach we are only given the final image.
Online recognition is primarily used for signature verification, author authentication and digital pens.
Application areas for offline recognition include postal automation, bank cheque processing and automatic data entry \cite{intro1}.
Formally, handwritten recognition is the task of transforming a language represented in graphical form into its symbolic representation \cite{introsurvey}.

The ultimate goal in handwritten recognition is to recognize words.
However, one way to potentially decompose or simplify the problem is to segment words into its individual characters \cite{intro-Yacoubi}. 
Segmentation can either be done \textit{explicitly} or \textit{implicitly}.
Explicit segmentation tries to separate the word at character boundaries while implicit segmentation separates the word into equal sized frames.
The implicit frames, each represented by a feature vector, are then mapped into characters.

This paper is focused on offline handwritten recognition.
We attempt to tackle both character and word recognition.
To simplify the word recognition problem, we assume that the images containing the words have already been explicitly segmented into new images containing the separated characters.
During doing word recognition, we also assume that the words come from a finite lexicon.
These assumptions were made due to the limited time available for this project.
In the following sections we describe how a handwritten text recognition system was developed based on Hidden Markov Models and evaluate it's performance.

\section{Previous work}
Because handwritten recognition is such a well-researched area there is a wealth of literature available.
We mention only a few references that we found helpful.
Cheriet et al. \cite{Cheriet} gives a good review of the development of handwritten recognition.
They also go on to give a broad overview of feature extraction and classification using a plethora of different techniques.
Rabiner, L. R. \cite{Rabiner1989} gives an excellent review of HMMs and the Baum-Welch training algorithm, as well as how to apply them in speech recognition.
El-Yacoubi et al. \cite{intro-Yacoubi} introduce an approach to recognize text using Hidden Markov Models with explicit word segmentation.
Laan et al. \cite{initialmodel} evaluate three different initial model selection techniques for the Baum-Welch algorithm, randomized, uniform and count-based initialization.
But despite impressive progress over the last couple of decades, performance is still far away from human performance.




\section{Previous work}
%Write about other papers and how they have solved the problem.

\section{Problem}
%Describe our limited version of the problem.
%Describe the two problems, one is recognizing characters and the other is to recognize words.
%1. Images have strokes of width 1.
%2. Assume characters segment in word.

\section{Method}\label{sec:method}
%2. General overview of our implementation.   Kjell
%	Two classifiers
%		What they are doing
%3. Add picture describing implementation.  Kjell

%1. Describe HMM, short overview. Chongyang
%1 Picture of HMM topology
%4. Describe the different initialization algos, advantages and disadvantages. Chongyang
%1.1Topology of HMM Chongyang
%1.2 prevention of underflow. Chongyang
%1.3 Handle zeros in denominator. Chongyang 




\subsection{Dataset}\label{sec:dataset}
%1. Write about creation of sample data. Chongyang
%2. Could not find good dataset. Chongyang
%3. Write down how much data we had. Chongyang
%4. Write about how the word examples are generated

\subsection{Preprocessing}
%1. Write about scaling of picture. Kjell
%2. Segmentation of picture. Kjell

\section{Result}\label{sec:result}
%1. (Started Kjell) Test of different initialization methods and number of training examples for word recognition 
%1. (Started Kjell) Test of different initialization methods and number of training examples for word recognition
%1. Maybe result from character test using different initialization algos.
%2. Basic results from testing the implementation
%3. (Better to do in appendix? Kjell) Depending on how much other results we have, discuss experience of playing around with the gui.

\subsection{Character Classification with Different Parameters}\label{sec:character_classifier_results}
In this section we describe results obtained with the \textit{character classifier}.
As described in Section~\ref{sec:image-preprocessing} the image feature extraction step in the character classifier takes two parameters.
The first parameter is the \textit{number of segments} that should be created. 
The second parameter is the \textit{size classification factor}, which is used in Equation~\ref{eq:classification_function}. 
For the experiment we only had 100 examples for each of the 26 characters. 
How the examples are produced is described in Section~\ref{sec:dataset}. 
An initial experiment was performed to test count-based initialization and random initialization before and after training with the Baum-Welch algorithm.
The initial experiment shows that there is probably not enough training data for the training to have any positive effect when using count-based initialization. 
This could possible be fixed to some extent with some kind of smoothening of the model parameters. 
10 test examples and 90 training examples for every character were selected randomly from the example set for the experiments. 
The results from the initial experiment can be found in Table~\ref{tab:character_classifier_initial_experiment}. 
In the initial experiment $1.3$ was used as the size classification factor and the number of segments was set to $7$.


\begin{table}[htb]
  \begin{center}
  \begin{tabular}{ l l l l l }
    NOE    & RIBF   & CBIBT  & RIAT    & CBIAT \\ \hline
    $90$  & $4\%$ & $53\%$ & $16\%$  & $16\%$  \\   
  \end{tabular}
\end{center}
\caption{Test of the character classifier with different initialization methods and before and after training.
	 NOE=''number of training examples for every word'',
         RIBF=''random initialization score before training'',
         CBIBT=''count-based initialization score before training'',
         RIAT=''random initialization score after training'',
         CBIAT=''count-based initialization score after training''} 
\label{tab:character_classifier_initial_experiment} 
\end{table}

Only count-based initialization is considered in the experiment of different parameters, because the initial experiment showed that the best result seems to be produced when only using count-based initialization and no training.
When testing the parameters, 5 models were created in the same way as in the initial experiment. 
The average accuracy for these 5 models when testing them with their own test example set was recorded as the accuracy for the configuration. 
For all 5 models that were created, different training example sets and test example sets were randomly selected. 
We used 90 training examples and 10 test examples for every character, as in the initial experiment. 
The results of the experiment can be seen in Figure~\ref{figure:charater-results-parameters}.

\begin{figure}[h!]
\centering
\includegraphics[scale=0.57]{ccf-nos}
\caption{Results for character classification test with different parameters. The performance is the percentage of correctly classified characters.}
\label{figure:charater-results-parameters}
\end{figure}

\subsection{Forward-classifier with Different Initialization Methods}\label{sec:word_classifier_results}
In this section we will describe results obtained for the \emph{Forward-classifier}, used to classify words, introduced in Section~\ref{sec:overview-of-classifiers}.
The classification results for running with the words in Table~\ref{tab:words_supported_by_classifier} will be presented. 
The training and test example words were randomly generated with the generator having the properties in Table~\ref{tab:word_generator_properties}.
See Section~\ref{sec:dataset}, for more information about the word example generator. 
We used a word example generator because it meant that we could generate as much training data as we needed.

We used a total of 100 test examples to test the accuracy of the created classifiers.
Five test examples each for the 20 words. 
The test examples were generated using the same properties as the training examples.
Two initialization methods, count-based initialization and random initialization, were tested with 100, 200, 400, 800 and 1600 training examples. 
The results of the test is presented in Figure~\ref{figure:initialization-methods}. 
It contains the test scores for the two initialization methods, before and after training with the Baum-Welch algorithm. 
The test score is defined as the percentage of correctly classified test examples.

\begin{table}[htb]
  \begin{center}
  \begin{tabular}{ l l l l l }
    dog      & cat       & pig     & love       & hate  \\
    scala    & python    & summer  & winter     & night  \\ 
    daydream & nightmare & animal  & happiness  & sadness \\ 
    tennis   & feminism  & fascism & socialism  & capitalism \\
  \end{tabular}
\end{center}
\caption{Words supported by the resulting classifier.} 
\label{tab:words_supported_by_classifier} 
\end{table}

\begin{table}[htb]
  \begin{center}
  \begin{tabular}{ l l }
    Probability of extra letter at position         & 0.03 \\
    Probability of extra letter equal neighbor      & 0.7 \\ 
    Probability of wrong letter at position         & 0.1 \\ 
    Probability of letter missing at position       & 0.03 \\
  \end{tabular}
\end{center}
\caption{Properties enforced by the word training example generator.} 
\label{tab:word_generator_properties} 
\end{table}

\begin{figure}[h!]
\centering
\includegraphics[scale=0.57]{initialization-methods}
\caption{Test with different number of training examples and different initialization methods.
	 NOE=''number of training examples for every word'',
         RIBF=''random initialization score before training'',
         CBIBT=''count-based initialization score before training'',
         RIAT=''random initialization score after training'',
         CBIAT=''count-based initialization score after training.}
\label{figure:initialization-methods}
\end{figure}

\subsection{Two Level Classification with the Viterbi-classifier}
In this section we will describe results obtained for the \emph{Viterbi-classifier} together with the character classifier.
The classifiers are introduced in Section~\ref{sec:overview-of-classifiers}.
Thus unlike in Section~\ref{sec:word_classifier_results},  we will illustrate the results for a fully working two stage offline-handwritten recognition system.


A test string for a word $w$ used as input to the classifier in the test was generated in the following way:

\begin{enumerate}
 \item A test example set with examples of character images containing 20 images for every character. 
For every letter in the string a corresponding character image was chosen randomly.
 \item The selected character images were classified with a character classifier which used 100 training examples, 11 segments and a \textit{classification factor} of 4.6.
 \item The resulting string from step 2 is the final test example for the word $w$.
\end{enumerate}

 Two experiments were ran with two different sets of possible output words.
 These sets can be seen in appendix~\ref{app:examples_sets_viterbi_classifier}.
 \emph{Example set 1} contains 151 unconstrained English words chosen from common prefix and root word examples in the dictionary.
 \emph{Example set 2} contains 8 words with 3 letters in each.
 To test the difference between using the Viterbi correction and simply the similarity measurement, we performed tests with and without the Viterbi correction.
 For every word in the list 10 test examples were created.
 With the Viterbi correction the score was 91\% for \emph{example set 1} and 96\% for \emph{example set 2}.
 Without the Viterbi correction the score was 91\% for \emph{example set 1} and 70\% for \emph{example set 2}.

 The results for the \emph{example set 1} are almost the same, with and without the viterbi step respectively.
 Figure~\ref{fig:viterbifig} shows how the string looks like before and after the viterbi step, as well as after the similarity mapping, in the this order.
 From the Figure it is possible to see that the Viterbi step actually improves the input string.

    \begin{figure}[htb] 
      \begin{center}
	\leavevmode
	\includegraphics[width=80mm]{viterbiimage.pdf}%width=115mm,height=40mm
      \end{center}
      \caption{Example image from output of the Viterbi-classifier experiment.}
      \label{fig:viterbifig}
    \end{figure}








\section{Discussion}
%1. Discuss the results.
%2. Discuss importance of amount of training data and the effects on performance.
%3. Discuss possibility of training words with data from character classification
%4. Discuss how our approach might work for word recognition

\section{Future Work}
%1. Discuss what other test would be interesting to perform
%2. How can the classifier be improved. More features and real training data for the words?

\bibliographystyle{plain}	% (uses file "plain.bst")
\bibliography{myrefs}		% expects file "myrefs.bib"

\appendix

\section{Reproduce Results}
%Kjell Describe how to run the tests
\section{Testing Handwriting Recognition in Graphical User Interface}
%Kjell Describe the GUI for running the HandReco Writer and perhaps some comments on how it works
\end{document}