\documentclass[11pt]{article}
\usepackage[utf8]{inputenc}
\usepackage{amsmath, amsthm, amssymb}
\usepackage{graphicx}
\author{Fabian Alenius, Kjell Winblad and Chongyuang Sun} \title{Handwritten Character Recgonition}
\begin{document}
\maketitle

\begin{abstract}
Hello.

\end{abstract}

\section{Introduction}
1. Give general introduction to area.
2. Describe why it's important.
3. Applications.
4. Contributions.

\cite{trec}
Fabian
\section{Introduction}

Recognition of handwritten text has been a popular research area for decades because it can be used in so many different applications.
There are two different approaches to handwritten recognition, \textit{online} and \textit{offline}.
In the online approach we know the order in which the strokes and individual points were drawn.
This information can easily be captured if the text is recorded by a digital pen or on a touchscreen.
In the offline approach we are only given the final image.
Online recognition is primarily used for signature verification, author authentication and digital pens.
Application areas for offline recognition include postal automation, bank cheque processing and automatic data entry \cite{intro1}.
Formally, handwritten recognition is the task of transforming a language represented in graphical form into its symbolic representation \cite{introsurvey}.

The ultimate goal in handwritten recognition is to recognize words.
However, one way to potentially decompose or simplify the problem is to segment words into its individual characters \cite{intro-Yacoubi}. 
Segmentation can either be done \textit{explicitly} or \textit{implicitly}.
Explicit segmentation tries to separate the word at character boundaries while implicit segmentation separates the word into equal sized frames.
The implicit frames, each represented by a feature vector, are then mapped into characters.

This paper is focused on offline handwritten recognition.
We attempt to tackle both character and word recognition.
To simplify the word recognition problem, we assume that the images containing the words have already been explicitly segmented into new images containing the separated characters.
During doing word recognition, we also assume that the words come from a finite lexicon.
These assumptions were made due to the limited time available for this project.
In the following sections we describe how a handwritten text recognition system was developed based on Hidden Markov Models and evaluate it's performance.

\section{Previous work}
Because handwritten recognition is such a well-researched area there is a wealth of literature available.
We mention only a few references that we found helpful.
Cheriet et al. \cite{Cheriet} gives a good review of the development of handwritten recognition.
They also go on to give a broad overview of feature extraction and classification using a plethora of different techniques.
Rabiner, L. R. \cite{Rabiner1989} gives an excellent review of HMMs and the Baum-Welch training algorithm, as well as how to apply them in speech recognition.
El-Yacoubi et al. \cite{intro-Yacoubi} introduce an approach to recognize text using Hidden Markov Models with explicit word segmentation.
Laan et al. \cite{initialmodel} evaluate three different initial model selection techniques for the Baum-Welch algorithm, randomized, uniform and count-based initialization.
But despite impressive progress over the last couple of decades, performance is still far away from human performance.




\section{Previous work}
Write about other papers and how they have solved the problem.

\section{Problem}
Describe our limited version of the problem.
Describe the two problems, one is recognizing characters and the other is to recognize words.
1. Images have strokes of width 1.
2. Assume characters segment in word.

\section{Method}
2. General overview of our implementation.   Kjell
	Two classifiers
		What they are doing
3. Add picture describing implementation.  Kjell

1. Describe HMM, short overview. Chongyang
1 Picture of HMM topology
4. Describe the different initialization algos, advantages and disadvantages. Chongyang
1.1Topology of HMM Chongyang
1.2 prevention of underflow. Chongyang
1.3 Handle zeros in denominator. Chongyang 




\subsection{Dataset}\label{sec:dataset}
1. Write about creation of sample data. Chongyang
2. Could not find good dataset. Chongyang
3. Write down how much data we had. Chongyang

\subsection{Preprocessing}
1. Write about scaling of picture. Kjell
2. Segmentation of picture. Kjell

\section{Result}\label{sec:result}
1. Maybe result from character test using different initialization algos.
2. Basic results from testing the implementation
3. Depending on how much other results we have, discuss experience of playing around with the gui.

\section{Discussion}
1. Discuss the results.
2. Discuss importance of amount of training data and the effects on performance.

\section{Future Work}
1. 

\bibliographystyle{plain}	% (uses file "plain.bst")
\bibliography{myrefs}		% expects file "myrefs.bib"
\end{document}